\documentclass[dvipdfmx,a4paper,10pt]{jarticle}
\usepackage{graphicx}
\usepackage{amsmath,amssymb}
\usepackage{mathtools}
\usepackage{bm}
\usepackage{latexsym}
\usepackage{multirow}
\usepackage{url}
\usepackage[separate-uncertainty]{siunitx}
\usepackage{physics}
\usepackage{pdfpages}
\usepackage{tcolorbox}
\usepackage{siunitx}
\tcbuselibrary{raster,skins}
\setlength{\textwidth}{165mm} %165mm-marginparwidth
\setlength{\marginparwidth}{40mm}
\setlength{\textheight}{225mm}
\setlength{\topmargin}{-5mm}
\setlength{\oddsidemargin}{-3.5mm}

\def\vector#1{\mbox{\boldmath $#1$}}
\newcommand{\AmSLaTeX}{%
 $\mathcal A$\lower.4ex\hbox{$\!\mathcal M\!$}$\mathcal S$-\LaTeX}
\newcommand{\PS}{{\scshape Post\-Script}}
\def\BibTeX{{\rmfamily B\kern-.05em{\scshape i\kern-.025em b}\kern-.08em
 T\kern-.1667em\lower.7ex\hbox{E}\kern-.125em X}}
\newcommand{\DeLta}{{\mit\Delta}}
\renewcommand{\d}{{\rm d}}
\def\wcaption#1{\caption[]{\parbox[t]{100mm}{#1}}}
\def\rm#1{\mathrm{#1}}
\def\tempC{^\circ \rm{C}}

\makeatletter
%\def\section{\@startsection {section}{1}{\z@}{-3.5ex plus -1ex minus % -.2ex}{2.3ex plus .2ex}{\Large\bf}}
\def\section{\@startsection {section}{1}{\z@}{-3.5ex plus -1ex minus
-.2ex}{2.3ex plus .2ex}{\normalsize\bf}}
\makeatother

\makeatletter
\def\subsection{\@startsection {subsection}{1}{\z@}{-3.5ex plus -1ex minus
-.2ex}{2.3ex plus .2ex}{\normalsize\bf}}
\makeatother

\makeatletter
\def\@seccntformat#1{\@ifundefined{#1@cntformat}%
   {\csname the#1\endcsname\quad}%      default
   {\csname #1@cntformat\endcsname}%    enable individual control
}
\makeatother

\newcommand{\tenexp}[2]{#1\times10^{#2}}


\begin{document}
\begin{center}
	{\Large{\bf ブラックホールを作ろう!}} \\
	{\bf{電気通信大学1年 国分 梓}}
\end{center}

\begin{tcolorbox}[enhanced,empty,left skip=0pt,left=5pt,
		coltitle=white,title={\bf ブラックホールって何?},sharp corners,
		overlay={
				\begin{tcbclipframe}
					\fill[black] (title.south west)--++(4.3,0)--++(0,1)--++(-4.3,0)--cycle;
				\end{tcbclipframe}}]
	ブラックホールとは何でしょうか.そうです,光でも何でも吸い込んでしまうアレです.今回はそんなブラックホールを作る方法をご紹介します.\\
	ここでは,イメージしやすいように太陽を用います.太陽は半径が約$6.9\times10^8\,\si{m}$,重さが約$1.9\times10^{30}\,\si{kg}$の天体です.これをブラックホールにしてみましょう.
\end{tcolorbox}
\begin{tcolorbox}[enhanced,empty,left skip=0pt,left=5pt,
		coltitle=white,title={\bf 作る},sharp corners,
		overlay={
				\begin{tcbclipframe}
					\fill[black] (title.south west)--++(1.3,0)--++(0,1)--++(-1.3,0)--cycle;
				\end{tcbclipframe}}]
	では実際に作ってみましょう.まず,太陽を半径が約3\si{km}になるまで圧縮します.はい,ブラックホールが完成しました.
\end{tcolorbox}

\begin{tcolorbox}[enhanced,empty,left skip=0pt,left=5pt,
		coltitle=white,title={\bf どういうこと?},sharp corners,
		overlay={
				\begin{tcbclipframe}
					\fill[black] (title.south west)--++(3,0)--++(0,1)--++(-3,0)--cycle;
				\end{tcbclipframe}}]
	これは一体どういうことでしょうか.式を用いて説明します.少し話が難しくなるかもしれませんが頑張りましょう!\\
	ある天体とその周囲を運動する物体からなる系を考えます.万有引力定数を$G$,物体の重さを$m$,運動の速さを$v$,天体の重さを$M$,天体と物体の距離を$r$としたとき,この系の力学的エネルギーは
	\begin{gather*}
		\frac{1}{2}mv^2 - G\frac{Mm}{r}
	\end{gather*}
	と表せます.ここで,物体が天体の重力を振り切って無限遠方に飛んでいくために必要な速度を脱出速度と呼びます.この脱出速度は,力学的エネルギーが0のときの速度$v$です.これより,上式を変形すると
	\begin{gather*}
		\frac{1}{2}mv^2 - G\frac{Mm}{r} = 0\\
		\therefore r = \frac{2GM}{v^2}
	\end{gather*}
	この世で最も速いのは光です.最も速い光ですら脱出できない,すなわち脱出速度が光より速くなる天体がブラックホールとなります.光の速さを$c$としたとき,ブラックホールの半径$R_s$は
	\begin{equation}
		R_s = \frac{2GM}{c^2}
		\nonumber
	\end{equation}
	と表されます.この式を用いて,実際に万有引力定数$G$,太陽の重さ$M$,光速$c$を代入して計算すると$R_s \fallingdotseq 3\rm{km}$となります.ちなみにこの$R_s$はシュワルツシルト半径と呼ばれています.
	
	\begin{tcolorbox}[enhanced,empty,left skip=0pt,left=5pt,
			coltitle=white,title={\bf どうやって圧縮するの?},sharp corners,
			overlay={
					\begin{tcbclipframe}
						\fill[black] (title.south west)--++(4.3,0)--++(0,1)--++(-4.3,0)--cycle;
					\end{tcbclipframe}}]
		頑張る.頑張って圧縮しましょう!諦めなければいつかはできるようになるはずです.\\
		以上,ブラックホールの作り方でした.お読み頂きありがとうございました!
	\end{tcolorbox}
	
\end{tcolorbox}

\end{document}
