\documentclass[a4paper,10pt]{jarticle}
\usepackage[dvipdfmx]{graphicx}
\usepackage[margin=30truemm]{geometry}
\pagestyle{empty}
%%%%%%%%%%%%%%%%%%%%%%%%%%%%%%%%%%%%%%%%%%%%%%%%%%%%%%%%%%%

\begin{document} 
\title{ごあいさつ}
\author{2年 部長 川口美玲}
\date{}
\maketitle

\begin{center}


\phantom{a}\par
本日は調布まち活フェスタ電気通信大学天文部のブースに足を運んでくださり、またこの部誌を手に取ってくださってありがとうございます。部長の川口美玲と申します。


\phantom{a}\par
電気通信大学天文部は個性豊かな29人の部員で活動しています。


\phantom{a}\par
高校時代も天文部に所属していて知識ばっちりの天文オタクな人、天体写真や星景写真を撮るのが得意な人、星の位置を計算している人、星座に詳しい人、なんとなく星空を眺めるのが好きな人...


\phantom{a}\par
それぞれがそれぞれの形で活動を楽しんでいます。この部誌にも個性あふれる記事が集まりました!


\phantom{a}\par
そんな個性豊かな部員が集まった天文部ですが、みんなで夜空を見上げているときは一体感を強く感じます。星空には不思議な力があるのでしょうか...私はその時間がとても好きです。


\phantom{a}\par
この部誌を読んで天文に興味を持っていただき、そしてぜひ皆様にも星空を眺めてみてほしいと思います。


\phantom{a}\par
皆様の天文ライフがより良いものとなりますように。


\phantom{a}\par
夜は星を見ナイト!


\end{center}
\end{document}
