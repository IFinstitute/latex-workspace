%! TEX root = ../../supernova_2023.tex
\documentclass[../../super_nova_2023]{subfiles}

\begin{document}
\chapter{ごあいさつ}
\vspace{2\zw}
\begin{center}


	\phantom{a}\par
	本日は調布まち活フェスタ電気通信大学天文部のブースに足を運んでくださり、またこの部誌を手に取ってくださってありがとうございます。部長の川口美玲と申します。
	
	
	\phantom{a}\par
	電気通信大学天文部は個性豊かな29人の部員で活動しています。
	
	
	\phantom{a}\par
	高校時代も天文部に所属していて知識ばっちりの天文オタクな人、天体写真や星景写真を撮るのが得意な人、星の位置を計算している人、星座に詳しい人、なんとなく星空を眺めるのが好きな人...
	
	
	\phantom{a}\par
	それぞれがそれぞれの形で活動を楽しんでいます。この部誌にも個性あふれる記事が集まりました!
	
	
	\phantom{a}\par
	そんな個性豊かな部員が集まった天文部ですが、みんなで夜空を見上げているときは一体感を強く感じます。星空には不思議な力があるのでしょうか...私はその時間がとても好きです。
	
	
	\phantom{a}\par
	この部誌を読んで天文に興味を持っていただき、そしてぜひ皆様にも星空を眺めてみてほしいと思います。
	
	
	\phantom{a}\par
	皆様の天文ライフがより良いものとなりますように。
	
	
	\phantom{a}\par
	夜は星を見ナイト!
	
	
\end{center}
\vspace{5\zw}
% \rightline{2年 部長 川口美玲}
\rightline{天文部長 2年 川口美玲~\includegraphics[height=2\zw]{cover/uec_astro_logo.png}}
\end{document}