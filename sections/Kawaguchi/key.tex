%! TEX root = ../supernova_2023.tex
\documentclass[supernova_2023]{subfiles}

\begin{document}

\article[1]{天文クロスワード}{2年 川口美玲}{sections/Kawaguchi/crossword.pdf}
\vspace{3\zw}
\begin{tcolorbox}[title=たてのカギ]
  \begin{description}
  \item[1. ] α星が北極星のポラリスである星座。〇〇〇座。
  \item[2. ] 太陽系の内側の地球外惑星。〇〇〇〇プラネット。
  \item[3. ] 月を肉眼で見たときやや暗く見える場所は海、明るく見える場所は〇〇と呼ばれる。
  \item[4. ] 電気通信大学にはプログラミング〇〇〇〇〇に参加している学生がよくいらっしゃいます。天文部の先輩方にも!?
  \item[5. ] 惑星、衛星、恒星などの核のこと。
  \item[6. ] ヒッパルコス衛星によって得られた1989~1993年にかけての観測データを使用して作られた。〇〇〇星表。
  \item[8. ] ユリウス暦で365日の平年に比べて1日多い366日ある年。〇〇〇年。
  \item[9. ] 惑星の軌道は太陽を1つの焦点とした楕円であることが第一法則とされている〇〇〇〇の法則。
  \item[12. ] 黒鉛、ダイアモンドを同素体にもつ元素。
  \item[13. ] ベテルギウスやリゲルをもち、中央に並んだ三ツ星が目印の〇〇〇〇座。
  \item[15. ] 天体と天の赤道との間の角距離。赤経と組み合わせて天体の位置を示す。
  \item[16. ] 宇宙飛行士の選抜試験では圧迫面接で宇宙飛行士としての〇〇〇があるか見極められる。
  \item[17. ] 三日月のこと。〇〇〇ムーン。
  \end{description}
\end{tcolorbox}
\vspace{3\zw}
\begin{tcolorbox}[title=よこのカギ]
  \begin{description}
    \item[1. ] 2022年10月から国際宇宙ステーションに滞在している日本人の若田〇〇〇〇宇宙飛行士。
    \item[5. ] 太陽表面にあり温度が低いために暗く見える構造。
    \item[7. ] すばる望遠鏡などが設置されているハワイ島北部に位置する標高4205mの山。
    \item[10. ] 11/23~12/21生まれの人の星座。〇〇座。
    \item[11. ] ある時間の流れを何度も繰り返す現象。タイム〇〇〇。
    \item[12. ] トルティーヤに具材を挟むメキシコ料理。SuperNova調布祭特別号にレシピが載っていました!こちら(\url{http://www.astro.club.uec.ac.jp/})から前号の部誌の電子版がご覧いただけます。
    \item[13. ] ドラゴンボールのキャラクターで全世界一強い神様、全〇〇。
    \item[14. ] みずがめ座にある惑星状星雲で中心部が猫の目のような形をしている〇〇〇星雲。
    \item[16. ] すみません、天文関係ございません。スペアリブをトッピングした沖縄の麵料理。おいしいですよね~!
    \item[18. ] 一等星のアルデバランをα星にもつ〇〇〇座。4/20~5/20生まれの人の星座。
    \item[19. ] アメリカ航空宇宙局の略称。
    \item[20. ] 地球との平均距離が約38万kmの地球を周回する衛星。
    \item[21. ] レゴリスとは月の〇〇のこと。
  \end{description}
\end{tcolorbox}
\vspace{3\zw}
{\footnotesize クロスワードの答えは巻末にあります.}
\end{document}