%! TEX root = ../supernova_2023.tex
\documentclass[supernova_2023]{subfiles}

\begin{document}
\chapter{天体と関係のあるポケモン紹介}
\rightline{2年 山際弘明}

\section{自己紹介\emoji{hugging-face}}
こんにちは,天文部副部長の山際弘明と申します.現在電気通信大学の$2$年生です.私は天文部のほかにも将棋部とUECポケモンだいすきクラブに所属しています.
せっかくなので将棋かポケモンに関する話を書きたいと思い,今回はポケモンに絡めて書くことにしました.最後まで読んでいただけると嬉しいです.
\section{ポケモンとは?}
ポケモンとはポケットモンスターの略称であり,\,$1996$年$2$月$27$日に任天堂から発売されたゲームからその歴史が始まりました.
現在ではゲームにとどまらず,アニメ・カードゲームなどで幅広い人から親しまれています.ポケモンにはたくさんの種類がありますが,
その中には天体に関係のあるポケモンもたくさんいます.全員を紹介することは紙面の都合上できないので,ほんの一部ですが紹介していきたいと思います.
\section{紹介}
\subsection{ルナトーン・ソルロック~\emoji{first-quarter-moon-face}・\emoji{sun-with-face}}
$2$匹とも「いんせきポケモン」に分類され,とくせいは「ふゆう」です.紹介文を見ると,それぞれ月と太陽に関係していると考えられます.
少し見た目がこわいかなと思うのですが,よく見るとかわいく思えてきませんか?
\subsection{アルセウス~\emoji{unicorn}}
「そうぞうポケモン」に分類され,とくせいは「マルチタイプ」です.そうぞうポケモンの「そうぞう」とは「創造」のことであると考えられ,
宇宙を創造したポケモンだといわれています.かっこいいですね.
\subsection{ジラーチ~\emoji{glowing-star}}
「ねがいごとポケモン」に分類され,とくせいは「てんのめぐみ」です.\,$1000$年のうち$7$日だけ姿を見せて,
どんなねがいごともかなえる力を使うといいます.どんなねがいごともかなえるといえば何を思い浮かべますか?そうです,流れ星ですね.
流れ星が流れている間に$3$回ねがいごとを唱えるとその願いが叶うといわれています.ロマンチックで素敵ですよね.
ジラーチが出てくる映画もあるので,興味を持った方はぜひ見てみてください.
\section{おわりに}
いかがでしたか?今回紹介したポケモン以外にも,天体にまつわるポケモンはたくさんいます.これを読んだ皆さんにもぜひ調べていただき,
天体とポケモンに興味を持っていただけたら大変うれしいです.
\section{参考文献}
ポケモンの紹介文,およびイラスト\footnote{編集部注)権利上,画像の掲載をしておりません.あらかじめご了承ください.}はポケモンずかん(\url{https://zukan.pokemon.co.jp})を参考にしました.
全てのポケモンを調べることができるので,ポケモンを調べたいときにはぜひご利用ください.
% \newpage
% あいているところには天体と関係のあるポケモンや好きなポケモンをかいてみよう!
% \vspace{2mm}
% \begin{figure}[H]
%  \includegraphics[scale=0.4]{pokemon1.png}
% \end{figure}
% \begin{figure}[H]
%  \includegraphics[scale=0.4]{polemon2.png}
% \end{figure}
\end{document}
