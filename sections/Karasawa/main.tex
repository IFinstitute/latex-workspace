%! TEX root = ../../supernova_2023.tex
\documentclass[../../super_nova_2023]{subfiles}

\begin{document}

\chapter{調布周辺のプラネタリウム}
\rightline{2年 唐澤香梨菜}

こんにちは。私は天体や写真に詳しくないので、調布周辺のプラネタリウムについてまとめてみました。プラネタリウムが見たくなった人のおでかけの参考になればいいなと思います。

\begin{enumerate}
	\item \phantomsection
	      \addcontentsline{toc}{section}{府中市郷土の森博物館}府中市郷土の森博物館\mbox{}\\約14万平方メートルの敷地全体で府中の自然・地形・風土の特徴を表現している府中郷土の森博物館は、2Fの常設展は府中の歴史や自然をテーマとしていて、1Fにはプラネタリウムと天文展示コーナーがあります。プラネタリウムは府中市の会社が制作したもので、より本物に近い星空の再現にこだわっています。すべての番組にその日の星空の生解説があり、迫力ある星空や映像を楽しむことができます。
	      \tcbox[left=0mm, right=0mm, top=0mm, bottom=0mm, boxsep=0mm, toptitle=0.5mm, bottomtitle=0.5mm, title=料金(個人)]{
		      \begin{tabular}{c|c|c}
			                 & 大人   & 中学生以下 \\ \hline
			      入場料        & 300円 & 150円  \\ \hline
			      プラネタリウム観覧料 & 600円 & 300円  \\
		      \end{tabular}
	      }
	      \vspace{-\zw}アクセス: 調布駅から車で20分、電車と徒歩で30分ほど
	      \vspace{\zw}
	\item \phantomsection
	      \addcontentsline{toc}{section}{多摩六都科学館}多摩六都科学館\mbox{}\\小平市、東村山市、清瀬市、東久留米市、西東京市の5つの市が運営しているプラネタリウム、観察、実験、工作が楽しめる体験型ミュージアムです。プラネタリウムは直径27.5mのドームに1億4000万個の星々を映し出し、スタッフによる生解説があり子供から大人まで幅広い世代で楽しむことができます。
	      \tcbox[left=0mm, right=0mm, top=0mm, bottom=0mm, boxsep=0mm, toptitle=0.5mm, bottomtitle=0.5mm, title=料金]{
		      \begin{tabularx}{80mm}{X|c|c}
			                                                       & 大人    & 4歳~高校生 \\ \hline
			      入館券(展示室)                                         & 520円  & 210円   \\ \hline
			      観覧付き入館券\par {\scriptsize (展示室+プラネタリウムまたは大型映像1回)} & 1040円 & 420円   \\ \hline
			      セット券\par {\scriptsize (展示室+プラネタリウム+大型映像)}        & 1460円 & 530円   \\
		      \end{tabularx}
	      }
	      \vspace{-\zw}アクセス: およそ調布駅から車で45分、電車とバスで1時間15分
	      \vspace{\zw}
	\item \phantomsection
	      \addcontentsline{toc}{section}{かわさき宙と緑の科学館}かわさき宙と緑の科学館\mbox{}\\1F展示室では南北に長い川崎市の自然を5つのテーマに分けて展示しています。また1Fと2Fにある天文展示では太陽系の8個の惑星の写真や、直接触ることもできる実物の隕石の展示、様々な星雲・星団、銀河系、宇宙の構造の解説があります。
	      プラネタリウムでは毎月変わる科学館のオリジナル番組を、その時々に観察できる天文現象の紹介とともに、専任の解説員の生解説を聞くことができます。 
	      \tcbox[left=0mm, right=0mm, top=0mm, bottom=0mm, boxsep=0mm, toptitle=0.5mm, bottomtitle=0.5mm, title=料金]{
		      \begin{tabular}{c|c|c|c}
			                 & 一般   & 高校生・大学生・65歳以上の方 & 中学生以下 \\ \hline
			      入場料        & 無料   & 無料              & 無料    \\ \hline
			      プラネタリウム観覧料 & 400円 & 200円            & 無料    \\
		      \end{tabular}
	      }
	      \vspace{-\zw}アクセス: 調布駅から車で30分、電車とバスで1時間ほど
	      \vspace{\zw}
	\item \phantomsection
	      \addcontentsline{toc}{section}{国立天文台}国立天文台\mbox{}\\調布にお住いの皆様なら知っているであろう国立天文台は、日本の天文学の中核を担う研究機関です。本部の三鷹キャンパスでは施設を自由に見学することができ、春休み期間中は第一赤道儀室での太陽観察会を実施している日もあります。
	      また4次元デジタル宇宙プロジェクト(4D2Uプロジェクト)では、天体や天体現象に時間も加えた4次元で「4次元デジタル宇宙コンテンツ」を作成し、スーパーコンピュータや最新の観測装置から得られるデータを科学的に可視化した立体映像を楽しむことができます。月3回の定例公開は各回定員20名ですが、国立天文台以外でも科学技術館や日本科学未来館などで公開されています。
	      \tcbox[left=0.5mm, right=0.5mm, top=0.5mm, bottom=0.5mm, boxsep=0mm, toptitle=0.5mm, bottomtitle=0.5mm, title=料金]{
		      入場料,観覧料ともに無料
	      }
	      \vspace{-\zw}アクセス: 調布駅から車で10分、バスで20分ほど
	      \vspace{\zw}
	\item \phantomsection
	      \addcontentsline{toc}{section}{コニカミノルタ}コニカミノルタ\mbox{}\\光学式・デジタル投影機、LEDドーム開発や製造などを行うコニカミノルタプラネタリウムが運営するプラネタリウムです。今まで紹介したのは博物館などのプラネタリウムでしたが、ここはエンターテイメント重視という印象を受けます。有楽町・池袋・東京スカイツリータウン・横浜・名古屋で上映していて、一人でふらっと行って癒しを求めたり、大切な人と特別な時間を過ごしたり、おしゃれなグッズやカフェで銀河を感じたりすることができます。
	      \tcbox[left=0.5mm, right=0.5mm, top=0.5mm, bottom=0.5mm, boxsep=0mm, toptitle=0.5mm, bottomtitle=0.5mm, title=料金]{
		      上映場所・プログラムにより異なりますが、大人であるとおよそ1500~4000円くらいです
	      }
	      \vspace{-\zw}アクセス: 調布駅から一番近い池袋まで車で1時間、電車で50分ほど
\end{enumerate}

\phantomsection
\addcontentsline{toc}{section}{各プラネタリウムのHP}
\section*{各プラネタリウムのHP}
これらはあくまで私が調べた情報なので実際に観にいかれる時は再度調べてみてください。
ここまで読んでいただきありがとうございました。

\begin{description}
	\item[府中市郷土の森博物館: ] \url{http://www.fuchu-cpf.or.jp/museum/}
	\item[多摩六都科学館: ] \url{https://www.tamarokuto.or.jp/}
	\item[かわさき宙と緑の科学館: ] \url{ https://www.nature-kawasaki.jp/}
	\item[国立天文台: ] \url{https://www.nao.ac.jp/}
	\item[コニカミノルタ: ] \url{https://planetarium.konicaminolta.jp/}
\end{description}
\end{document}