%! TEX root = ../supernova_2023.tex
\documentclass[supernova_2023]{subfiles}

\begin{document}
\chapter{編集後記}
\vspace{2\zw}
春の暖かな陽気が徐々に感じられるようになってきました.4月には新しい電通大生が仲間入りする新歓期がやってきます.暖かい季節なら活動しやすく,初心者でも比較的軽装で観測会ができるので,新歓期にはもってこいです.しかし,暖かくなると同時に天気はあまり安定せず,肝心の夜に晴れてくれません.天気予報では晴れの予報でも,夜は完全に曇っているなんてことは日常です.是非とも夜も含めて天気予報して頂きたいものです.それはさておき,晴れた日には観測会ができます.都市光害のさなか敢えて天体観測を楽しもうとする電通大天文部ですが,春から夏にかけての季節は苦手としています.晴れてもあまりなにも見えないのです.いや,毎度のごとくよく目を凝らせば見えるんです.見えるんですが,見える物と言えば北斗七星からアルトゥールスを通ってスピカへ続く春の大曲線だったりと,天文に詳しい人にはわかるようなものがギリギリ見える位で素人受けはあまりよくありません.春の星座はあまり明るくないので,調布で全体像を見るのは至難の業です.みんなが知ってるオリオン座やオリオン大星雲,アンドロメダ銀河,望遠鏡で覗く定番の土星や木星は全部夏以降でないと,大学が終わって放課後のサークル活動時間までに見ることができないのです.

2022年度はコロナ禍からの復活間もない頃でしたから,見かけのいい星が見えず,望遠鏡も満足に使えない状況で新歓活動に苦労しましたが,2023年度は違います.たとえ星が見えなくとも,今はプラネタリウムがあります.望遠鏡だって,今では大学の施設を利用しながら整備や修繕を行いかなりの台数が稼働できるようになっています.カメラを買う部員も増えて,天体写真や星景写真もたくさん集まりました.この部誌だってあります.天文部を説明するための材料はもう十分にそろっています.

少し今年の天文現象について話すと,2023年は三大流星群(1月しぶんぎ座,8月ペルセウス座,12月ふたご座)の時期が新月期と重なり,観測の条件がかなり良い年であると言えます.もちろん,調布で見てもかなりの量が見られると期待できますが,今年は少し郊外に出かけてみて見たい気もします.裏表紙に「東京だからこそ見える星空がある」なんて大層なこと書いちゃってますが,結局は暗い場所で見るのに勝る物はありません.たまの合宿に行く度に東京の空で満足できなくなる自分もいます.でもやっぱり,東京で撮った星空を後で編集して星をあぶり出していると,暗い場所で見るのとはまた違う快感もあったりするわけです.星空って奥が深いですね.

ここまでお読み頂きありがとうございました.もしここまで読んで電気通信大学天文部に興味を持ったというそこの君!電通大でお待ちしています.

\vspace{2\zw}
\rightline{部誌編集長 3年 森山陽介~\includegraphics[height=2\zw]{cover/uec_astro_logo.png}}
\end{document}

